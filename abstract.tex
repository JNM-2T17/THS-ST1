%%%%%%%%%%%%%%%%%%%%%%%%%%%%%%%%%%%%%%%%%%%%%%%%%%%%%%%%%%%%%%%%%%%%%%%%%%%%%%%%%%%%%%%%%%%%%%%%%%%%%%
%
%   Filename    : abstract.tex 
%
%   Description : This file will contain your abstract.
%                 
%%%%%%%%%%%%%%%%%%%%%%%%%%%%%%%%%%%%%%%%%%%%%%%%%%%%%%%%%%%%%%%%%%%%%%%%%%%%%%%%%%%%%%%%%%%%%%%%%%%%%%

\begin{abstract}
Social media has become a resource for data in the past years. Community detection in social media has become 
a popular field of study because of the large volume of data that could be extracted. This study aims to 
determine which combination of community detection algorithms and similarity parameters can produce the best
communities in Facebook and Twitter and to produce a visualization of the communities. Multiple iterations 
of community detection will be performed for each selected combination of algorithm and parameter. This study 
will also attempt to delve into the domain of Facebook community detection, a domain which has not been 
adequately explored. The results of this study can improve community detection in the future by determining 
the algorithm and parameter that would produce the best communities. This also means that, using the generated
visualization, meaningful data can already be extracted, which can be used by businesses or politicians to
determine patterns in their target audience.

\begin{flushleft}
\begin{tabular}{lp{4.25in}}
\hspace{-0.5em}\textbf{Keywords:}\hspace{0.25em} & Social networks, community detection, sentiment analysis\\
\end{tabular}
\end{flushleft}
\end{abstract}
