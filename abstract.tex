%%%%%%%%%%%%%%%%%%%%%%%%%%%%%%%%%%%%%%%%%%%%%%%%%%%%%%%%%%%%%%%%%%%%%%%%%%%%%%%%%%%%%%%%%%%%%%%%%%%%%%
%
%   Filename    : abstract.tex 
%
%   Description : This file will contain your abstract.
%                 
%%%%%%%%%%%%%%%%%%%%%%%%%%%%%%%%%%%%%%%%%%%%%%%%%%%%%%%%%%%%%%%%%%%%%%%%%%%%%%%%%%%%%%%%%%%%%%%%%%%%%%


\begin{abstract}
	In the past few years, social media has become an avenue for users to build relations with friends and share thoughts and opinions. These thoughts, opinions, and relations eventually give rise to communities of users, 
	which have more in common within the community relative to outside the community. These communities are not 
	readily visible in social media; community detection is necessary to extract them. Community detection in social media has become 
	a popular field of study because of the large volume and variety of data that could be extracted. This study aims to 
	determine which combination of community detection algorithms and similarity parameters can produce appropriate communities on Twitter and to produce a visualization of the communities. Multiple iterations 
	of community detection will be performed for each selected combination of algorithm and parameter. 
	This also means that, using the generated visualization, meaningful data can already be extracted, 
	which can be used by businesses or politicians to determine patterns in their target audience.
	
	
	\begin{flushleft}
		\begin{tabular}{lp{4.25in}}
			\hspace{-0.5em}\textbf{Keywords:}\hspace{0.25em} & Social networks, community detection, sentiment analysis\\
		\end{tabular}
	\end{flushleft}
\end{abstract}




