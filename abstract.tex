%%%%%%%%%%%%%%%%%%%%%%%%%%%%%%%%%%%%%%%%%%%%%%%%%%%%%%%%%%%%%%%%%%%%%%%%%%%%%%%%%%%%%%%%%%%%%%%%%%%%%%
%
%   Filename    : abstract.tex 
%
%   Description : This file will contain your abstract.
%                 
%%%%%%%%%%%%%%%%%%%%%%%%%%%%%%%%%%%%%%%%%%%%%%%%%%%%%%%%%%%%%%%%%%%%%%%%%%%%%%%%%%%%%%%%%%%%%%%%%%%%%%


\begin{abstract}
	In the past few years, social networking has become an avenue for users to build relations with friends and share thoughts and opinions. These in turn eventually give rise to communities of users. Communities of users are characterized as groups of users that have more in common with each other than they do with users outside of their communities. This commonality is measured by one or more similarity parameters, which are features of the network that serve to reflect the real-life interactions of these users. Through communities, the interactions of multiple users can be observed, and the large volume and variety of data on social networks can be organized in meaningful ways. However, current social networking platforms do not readily provide users with a means to view these communities. Community detection addresses this by providing ways to extract the communities present in these networks. This study aims to determine which combinations of community detection algorithms and similarity parameters can produce appropriate communities. A visualization of these communities will also be produced. To test the different combinations, multiple iterations of community detection will be performed, using data gathered from the Twitter social network. This also means that, using the generated visualization, meaningful data can be extracted from the network. This data can then be used by businesses or politicians to determine patterns in their target audiences.
	
	
	\begin{flushleft}
		\begin{tabular}{lp{4.25in}}
			\hspace{-0.5em}\textbf{Keywords:}\hspace{0.25em} & Social networks, community detection, sentiment analysis\\
		\end{tabular}
	\end{flushleft}
\end{abstract}




